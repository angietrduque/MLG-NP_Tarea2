% Options for packages loaded elsewhere
\PassOptionsToPackage{unicode}{hyperref}
\PassOptionsToPackage{hyphens}{url}
%
\documentclass[
]{article}
\usepackage{lmodern}
\usepackage{amssymb,amsmath}
\usepackage{ifxetex,ifluatex}
\ifnum 0\ifxetex 1\fi\ifluatex 1\fi=0 % if pdftex
  \usepackage[T1]{fontenc}
  \usepackage[utf8]{inputenc}
  \usepackage{textcomp} % provide euro and other symbols
\else % if luatex or xetex
  \usepackage{unicode-math}
  \defaultfontfeatures{Scale=MatchLowercase}
  \defaultfontfeatures[\rmfamily]{Ligatures=TeX,Scale=1}
\fi
% Use upquote if available, for straight quotes in verbatim environments
\IfFileExists{upquote.sty}{\usepackage{upquote}}{}
\IfFileExists{microtype.sty}{% use microtype if available
  \usepackage[]{microtype}
  \UseMicrotypeSet[protrusion]{basicmath} % disable protrusion for tt fonts
}{}
\makeatletter
\@ifundefined{KOMAClassName}{% if non-KOMA class
  \IfFileExists{parskip.sty}{%
    \usepackage{parskip}
  }{% else
    \setlength{\parindent}{0pt}
    \setlength{\parskip}{6pt plus 2pt minus 1pt}}
}{% if KOMA class
  \KOMAoptions{parskip=half}}
\makeatother
\usepackage{xcolor}
\IfFileExists{xurl.sty}{\usepackage{xurl}}{} % add URL line breaks if available
\IfFileExists{bookmark.sty}{\usepackage{bookmark}}{\usepackage{hyperref}}
\hypersetup{
  pdftitle={Tarea 2: Modelos lineales generalizados y paramétricos},
  pdfauthor={Angie Rodriguez Duque \& Cesar Saavedra Vanegas},
  hidelinks,
  pdfcreator={LaTeX via pandoc}}
\urlstyle{same} % disable monospaced font for URLs
\usepackage[margin=1in]{geometry}
\usepackage{color}
\usepackage{fancyvrb}
\newcommand{\VerbBar}{|}
\newcommand{\VERB}{\Verb[commandchars=\\\{\}]}
\DefineVerbatimEnvironment{Highlighting}{Verbatim}{commandchars=\\\{\}}
% Add ',fontsize=\small' for more characters per line
\usepackage{framed}
\definecolor{shadecolor}{RGB}{248,248,248}
\newenvironment{Shaded}{\begin{snugshade}}{\end{snugshade}}
\newcommand{\AlertTok}[1]{\textcolor[rgb]{0.94,0.16,0.16}{#1}}
\newcommand{\AnnotationTok}[1]{\textcolor[rgb]{0.56,0.35,0.01}{\textbf{\textit{#1}}}}
\newcommand{\AttributeTok}[1]{\textcolor[rgb]{0.77,0.63,0.00}{#1}}
\newcommand{\BaseNTok}[1]{\textcolor[rgb]{0.00,0.00,0.81}{#1}}
\newcommand{\BuiltInTok}[1]{#1}
\newcommand{\CharTok}[1]{\textcolor[rgb]{0.31,0.60,0.02}{#1}}
\newcommand{\CommentTok}[1]{\textcolor[rgb]{0.56,0.35,0.01}{\textit{#1}}}
\newcommand{\CommentVarTok}[1]{\textcolor[rgb]{0.56,0.35,0.01}{\textbf{\textit{#1}}}}
\newcommand{\ConstantTok}[1]{\textcolor[rgb]{0.00,0.00,0.00}{#1}}
\newcommand{\ControlFlowTok}[1]{\textcolor[rgb]{0.13,0.29,0.53}{\textbf{#1}}}
\newcommand{\DataTypeTok}[1]{\textcolor[rgb]{0.13,0.29,0.53}{#1}}
\newcommand{\DecValTok}[1]{\textcolor[rgb]{0.00,0.00,0.81}{#1}}
\newcommand{\DocumentationTok}[1]{\textcolor[rgb]{0.56,0.35,0.01}{\textbf{\textit{#1}}}}
\newcommand{\ErrorTok}[1]{\textcolor[rgb]{0.64,0.00,0.00}{\textbf{#1}}}
\newcommand{\ExtensionTok}[1]{#1}
\newcommand{\FloatTok}[1]{\textcolor[rgb]{0.00,0.00,0.81}{#1}}
\newcommand{\FunctionTok}[1]{\textcolor[rgb]{0.00,0.00,0.00}{#1}}
\newcommand{\ImportTok}[1]{#1}
\newcommand{\InformationTok}[1]{\textcolor[rgb]{0.56,0.35,0.01}{\textbf{\textit{#1}}}}
\newcommand{\KeywordTok}[1]{\textcolor[rgb]{0.13,0.29,0.53}{\textbf{#1}}}
\newcommand{\NormalTok}[1]{#1}
\newcommand{\OperatorTok}[1]{\textcolor[rgb]{0.81,0.36,0.00}{\textbf{#1}}}
\newcommand{\OtherTok}[1]{\textcolor[rgb]{0.56,0.35,0.01}{#1}}
\newcommand{\PreprocessorTok}[1]{\textcolor[rgb]{0.56,0.35,0.01}{\textit{#1}}}
\newcommand{\RegionMarkerTok}[1]{#1}
\newcommand{\SpecialCharTok}[1]{\textcolor[rgb]{0.00,0.00,0.00}{#1}}
\newcommand{\SpecialStringTok}[1]{\textcolor[rgb]{0.31,0.60,0.02}{#1}}
\newcommand{\StringTok}[1]{\textcolor[rgb]{0.31,0.60,0.02}{#1}}
\newcommand{\VariableTok}[1]{\textcolor[rgb]{0.00,0.00,0.00}{#1}}
\newcommand{\VerbatimStringTok}[1]{\textcolor[rgb]{0.31,0.60,0.02}{#1}}
\newcommand{\WarningTok}[1]{\textcolor[rgb]{0.56,0.35,0.01}{\textbf{\textit{#1}}}}
\usepackage{graphicx,grffile}
\makeatletter
\def\maxwidth{\ifdim\Gin@nat@width>\linewidth\linewidth\else\Gin@nat@width\fi}
\def\maxheight{\ifdim\Gin@nat@height>\textheight\textheight\else\Gin@nat@height\fi}
\makeatother
% Scale images if necessary, so that they will not overflow the page
% margins by default, and it is still possible to overwrite the defaults
% using explicit options in \includegraphics[width, height, ...]{}
\setkeys{Gin}{width=\maxwidth,height=\maxheight,keepaspectratio}
% Set default figure placement to htbp
\makeatletter
\def\fps@figure{htbp}
\makeatother
\setlength{\emergencystretch}{3em} % prevent overfull lines
\providecommand{\tightlist}{%
  \setlength{\itemsep}{0pt}\setlength{\parskip}{0pt}}
\setcounter{secnumdepth}{-\maxdimen} % remove section numbering

\title{Tarea 2: Modelos lineales generalizados y paramétricos}
\author{Angie Rodriguez Duque \& Cesar Saavedra Vanegas}
\date{Octubre - 2020}

\begin{document}
\maketitle

\hypertarget{actividad-1}{%
\section{Actividad 1}\label{actividad-1}}

\hypertarget{section}{%
\subsection{}\label{section}}

Se dispone de los tiempos de vida (tiempos hasta que fallan, en horas)
de \(49\) recipientes de presión sometidos a un nivel de carga del
\(70\%\)

\hypertarget{distribuciuxf3n-weibull}{%
\subsection{Distribución Weibull}\label{distribuciuxf3n-weibull}}

Para estudiar este tipo de variable se acostumbra utilizar la
distribución de Weibull, cuya función de densidad es:

\[f(y;\lambda,\theta)=\displaystyle\frac{\lambda y^{\lambda-1}}{\theta^{\lambda}}exp \left[-\left(\frac{y}{\theta}\right)^{\lambda}\right] \]

Para analizar este problema usaremos el método de Newton (conocido
también como el método de Newton Raphson o el método de Newton Fourier),
el cual es un algoritmo basado en la derivada que permite encontrar
aproximaciones de los ceros o raíces de una función real derivable. En
este caso particular se hara uso de la función U de scoring para la
Weibul y se asumira \(\lambda\) conocido y \(U\) será el estimador
\(\hat\theta\) del parametro de escala \(\theta\).

\begin{enumerate}
\def\labelenumi{\arabic{enumi}.}
\item
  Se elige \(x_{0}\) en el eje de las \(x\), asumiendo que está cerca de
  la solución de \(f(x)=0\) (raíz buscada)
\item
  Calculamos la ecuación punto pendiente de la recta tangente a la
  función en \((x_{0},f(x_{0}))\), a saber
  \(y−f(x_{0})=f′(x_{0})(x−x_{0})\) (1)
\item
  Esta recta debe intersecar al eje de las \(x\), en un punto más
  cercano a la raíz buscada; en el punto \((x_{1},0)\).
\item
  El punto \((x_{1},0)\) satisface la ecuación (1) y sustituyendo, queda
  (2): \[0−f(x0)=f′(x0)(x−x0)\]
\item
  Si \(f(x0)=0\), entonces despejando \(x_{1}\) en (2) se obtiene:
\end{enumerate}

\[x_{1}=x_{0}−\frac{f(x_{0})}{f′(x_{0})}\]

\begin{enumerate}
\def\labelenumi{\arabic{enumi}.}
\setcounter{enumi}{5}
\item
  Repetimos el procedimiento anterior para \(x_{0}\), pero ahora
  comenzando con \(x_{1}\), en cuyo caso se obtiene
  \(x_{2}=x_{1}−\frac{f(x_{1})}{f′(x_{1})}\). De forma que \(x_{2}\)
  está más cerca de la raíz buscada que \(x\).
\item
  Iterando cada vez con el número obtenido, se construye una secuencia:
  \(x_{0},x_{1},x−2,…,x_{n},…\) de números cada vez más próximos a la
  raíz, tales que: \[x_{n+1}=x_{n}−\frac{f(x_{n})}{f′(x_{n})}\]
\end{enumerate}

Siguiendo este algoritmo y realizando su representación en R se obtienen
los siguientes resultados para \(\lambda=2\):\#\# Función U de scoring
para la Weibull

\[ \frac{dl}{d\theta}=U=\displaystyle\sum_{i=1}^{N}\left[ \frac{-\lambda}{\theta}+\frac{\lambda y^{\lambda}_{i}}{\theta^{\lambda+1}}\right]\]
Como asumimos que conocemos \(\lambda\), la solución de \(U=0\) será el
estimador \(\hat{\theta}\) del parámetro de escala \(\theta\).

\hypertarget{funciuxf3n-u-de-scoring-para-la-weibull-con-lambda2}{%
\subsubsection{\texorpdfstring{Función U de scoring para la Weibull con
\(\lambda=2\)}{Función U de scoring para la Weibull con \textbackslash lambda=2}}\label{funciuxf3n-u-de-scoring-para-la-weibull-con-lambda2}}

Como en este caso estamos asumiendo que \(\lambda=2\), entonces la
expresión anterior será aplicable en este caso.

\[U=\frac{2\cdot 36}{\theta}+\frac{2\cdot \displaystyle\sum_{i=1}^{36}y^{2}_{i}}{\theta^{3}}\]
Nos proponemos encontrar \(\theta\) tal que \(U=0\)

\hypertarget{el-muxe9todo-de-newton-raphson}{%
\subsubsection{El método de
Newton-Raphson}\label{el-muxe9todo-de-newton-raphson}}

\[ \frac{dU}{d\theta}=U’=\displaystyle\sum_{i=1}^{N}\left[ \frac{\lambda}{\theta^{2}}+\frac{(\lambda) (\lambda-1) y^{\lambda}_{i}}{\theta^{\lambda+2}}\right]\]

\hypertarget{actividad-2}{%
\section{Actividad 2}\label{actividad-2}}

\hypertarget{base-de-datos}{%
\subsection{Base de datos}\label{base-de-datos}}

\begin{Shaded}
\begin{Highlighting}[]
\KeywordTok{dim}\NormalTok{(Datos)}
\end{Highlighting}
\end{Shaded}

\begin{verbatim}
## [1] 1599   12
\end{verbatim}

Este conjunto de datos de vino tinto consta de 1599 observaciones y 12
variables, 11 de las cuales son sustancias químicas. Las variables son:

\begin{enumerate}
\def\labelenumi{\arabic{enumi}.}
\item
  \textbf{Acidez fija:} La mayoría de los ácidos implicados en el vino
  son fijos o no volátiles (no se evaporan fácilmente).
\item
  \textbf{Acidez volátil:} La cantidad de ácido acético en el vino, que
  en niveles demasiado altos puede provocar un sabor desagradable a
  vinagre.
\item
  \textbf{Ácido cítrico:} Encontrado en pequeñas cantidades, el ácido
  cítrico puede agregar ``frescura'' y sabor a los vinos.
\item
  \textbf{Azúcar residual:} Es la cantidad de azúcar que queda después
  de que se detiene la fermentación, es raro encontrar vinos con menos
  de 1 gramo / litro y los vinos con más de 45 gramos / litro se
  consideran dulces.
\item
  \textbf{Cloruros:} Es la cantidad de sal del vino.
\item
  \textbf{Dióxido de azufre libre:} La forma libre de \(SO_{2}\) existe
  en equilibrio entre el \(SO_{2}\) molecular (como gas disuelto) y el
  ion bisulfito; Previene el crecimiento microbiano y la oxidación del
  vino.
\item
  \textbf{Dióxido de azufre total:} Es la cantidad de formas libres y
  unidas de \(SO_{2}\); en concentraciones bajas, el \(SO_{2}\) es
  mayormente indetectable en el vino, pero en concentraciones de
  \(SO_{2}\) libre superiores a 50 ppm, el \(SO_{2}\) se hace evidente
  en la nariz y el sabor del vino.
\item
  \textbf{Densidad:} La densidad es cercana a la del agua dependiendo
  del porcentaje de alcohol y contenido de azúcar.
\item
  \textbf{pH:} Describe qué tan ácido o básico es un vino en una escala
  de 0 (muy ácido) a 14 (muy básico); la mayoría de los vinos están
  entre 3-4 en la escala de pH.
\item
  \textbf{Sulfatos:} Aditivo del vino que puede contribuir a los niveles
  de dióxido de azufre \((SO_{2})\), que actúa como antimicrobiano y
  antioxidante.
\item
  \textbf{Alcohol:} El porcentaje de contenido de alcohol del vino.
\item
  \textbf{Calidad:} Variable de respuesta (basada en datos sensoriales,
  puntuación entre 0 y 10).
\end{enumerate}

\end{document}
